\section{Compilation}

This section deals with the more holistic side of the compilation process,
which transforms grammar into assembly.

\subsection{Named or Anonymous Grammars}

Naigama allows two types of grammars:

\begin{itemize}
\item Named grammars, which consist of a list of named rules.
\item Anonymous grammars, which consist of a single expression.
\end{itemize}

\subsection{The '\_\_prefix' Rule}

The \_\_prefix rule will define a pattern that is called before
the evaluation of every rule that follows it. This is typically
done to remove things like whitespace and comments while parsing
things like source code or text based data formats (XML, JSON, etc).

\subsection{The '\_\_main' Rule}

The '\_\_main' rule, which does not have to be defined, but if it is,
it will be called first.

\subsection{Determining the First Rule}

A named grammar will call as the first rule:

\begin{itemize}
\item The rule called '\_\_main', or in absence of such a rule:
\item The first defined rule.
\end{itemize}

The compiler will emit a 'call' instruction to the first rule as the
first instruction, followed by an 'end' instruction ('end 0').
