\subsection{Measure: Read-only-ness of Input and Bytecode}

The design of the modules that make up the Naigama parser system is such,
that inputs and bytecode are always considered read-only, allowing them to
be read from places that enforce this quality in something that is better
protected from modification than a kernel's flags on certain regions of
RAM.

Should it be necessary to make modifications to the input, then only in
the last module or stage of the process - the action processor - a copy
of the input can be made in read-write memory.

Read-onlyness of the memories used for storing inputs and bytecode, addresses
[Threat \thethreatbcupset] and [Threat \thethreatinput].
