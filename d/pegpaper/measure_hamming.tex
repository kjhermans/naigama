\subsection{Measure: Instruction Opcode Hamming Distance}

Naigama bytecode Instruction opcdes are chosen in such a manner
that it takes at least two bit flips for one instruction to be
possibly confused for another (Hamming Distance is always greater than,
or equal to two).

LPEG \cite{bib:lpeg} by contrast, uses a simple enum for the
opcode values [lpvm.h line 12].

[Table \ref{tab:naig_bytecode}] lists all Naigama parser
instructions with their opcode values.

The introduction of Hamming distance in Naigama bytecode instruction
opcode values addresses [Threat \thethreatbcupset].


%\begin{table}[]
\begin{center}
\caption{Naigama Bytecode Instructions}
\label{tab:naig_bytecode}
\begin{longtable}{lllll}
\textbf{Mnemonic} & \textbf{Opcode} & \textbf{Param1} & \textbf{Param2} & \textbf{Length} \\
\endhead
any & 000003e4 &  &   & 4 \\
backcommit & 000403c0 & address &   & 8 \\
call & 00040382 & address &   & 8 \\
catch & 00040393 & address &   & 8 \\
char & 000403d7 & char &   & 8 \\
closecapture & 00040300 & slot &   & 8 \\
commit & 00040336 & address &   & 8 \\
condjump & 00080321 & register & address  & 12 \\
counter & 00080356 & register & value  & 12 \\
end & 000400d8 & code &   & 8 \\
endisolate & 00003005 &  &   & 4 \\
endreplace & 00000399 &  &   & 4 \\
fail & 0000034b &  &   & 4 \\
failtwice & 00000390 &  &   & 4 \\
intrpcapture & 0008000f &  &   & 12 \\
isolate & 00043003 & slot &   & 8 \\
jump & 00040333 & address &   & 8 \\
maskedchar & 00080365 & char & mask  & 12 \\
mode & 0004000a &  &   & 8 \\
noop & 00000000 &  &   & 4 \\
opencapture & 0004039c & slot &   & 8 \\
partialcommit & 000403b4 & address &   & 8 \\
quad & 0004037e & quad &   & 8 \\
range & 000803bd & from & until  & 12 \\
replace & 00080348 & slot & address  & 12 \\
ret & 000003a0 &  &   & 4 \\
scr\_add & 0000050c &  &   & 4 \\
scr\_array & 00040006 &  &   & 8 \\
scr\_assign & 000005c9 &  &   & 4 \\
scr\_bitand & 00000527 &  &   & 4 \\
scr\_bitnot & 00000574 &  &   & 4 \\
scr\_bitor & 0000053c &  &   & 4 \\
scr\_bitxor & 0000052d &  &   & 4 \\
scr\_builtin & 000407cf &  &   & 8 \\
scr\_call & 00040503 &  &   & 8 \\
scr\_condjump & 0004000c &  &   & 8 \\
scr\_dec & 0000053a &  &   & 4 \\
scr\_div & 00000581 &  &   & 4 \\
scr\_equals & 0000056c &  &   & 4 \\
scr\_gt & 0000056f &  &   & 4 \\
scr\_gteq & 0000054d &  &   & 4 \\
scr\_inc & 000005f6 &  &   & 4 \\
scr\_index & 00000009 &  &   & 4 \\
scr\_logand & 0000052e &  &   & 4 \\
scr\_lognot & 000005f9 &  &   & 4 \\
scr\_logor & 000005a9 &  &   & 4 \\
scr\_lt & 00000595 &  &   & 4 \\
scr\_lteq & 00000522 &  &   & 4 \\
scr\_mul & 0000058b &  &   & 4 \\
scr\_nequals & 00000572 &  &   & 4 \\
scr\_pop & 000005cc &  &   & 4 \\
scr\_pow & 00000542 &  &   & 4 \\
scr\_push & 000c0003 &  &   & 16 \\
scr\_ret & 00000555 &  &   & 4 \\
scr\_shift & 00040005 &  &   & 8 \\
scr\_shiftin & 0000057d &  &   & 4 \\
scr\_shiftout & 00000517 &  &   & 4 \\
scr\_string & 000017bb &  &   & 4 \\
scr\_sub & 000005bb &  &   & 4 \\
set & 002003ca & set &   & 36 \\
skip & 00040330 & number &   & 8 \\
span & 002003e1 & set &   & 36 \\
testany & 00040306 & address &   & 8 \\
testchar & 0008039a & address & char  & 12 \\
testquad & 000803db & address & quad  & 12 \\
testset & 00240363 & address & set  & 40 \\
trap & ff00ffff &  &   & 4 \\
var & 000403ee & slot &   & 8 \\
\end{longtable}
\end{center}
%\end{table}


\subsection{Measure: Randomization of Opcode Values}

The source code base allows you to re-randomize the instruction set.
You can do this by issueing 'make instructions'. This will upset
your entire build tree and lock you out from previous versions
of instruction sets and bytecodes generated therein.

It is a roadmap item currently to generate a file format for bytecode
that comes with an 'instruction set signature' that allows engine
to recognize or reject non conforming bytecodes.
