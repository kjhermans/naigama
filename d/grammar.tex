\section{Grammar}

A Naigama grammar file or buffer consists either of a single
unnamed expression, or of a sequence of named expressions (called rules),
whitespace and comments, and denoted in ASCII text.
Naigama grammar is taken by the Naigama compiler program (naic) and turned
into assembly.

\begin{myquote}
\begin{verbatim}
-- JSON; JavaScript Object Notation.

TOP          <- JSON
__prefix     <- %s*
JSON         <- HASH END
HASH         <- CBOPEN OPTHASHELTS CBCLOSE
OPTHASHELTS  <- HASHELTS / ...
HASHELTS     <- HASHELT COMMA HASHELTS / HASHELT
HASHELT      <- STRING COLON VALUE
ARRAY        <- ABOPEN OPTARRAYELTS ABCLOSE
OPTARRAYELTS <- ARRAYELTS / ...
ARRAYELTS    <- VALUE COMMA ARRAYELTS / VALUE
VALUE        <- STRING / FLOAT / INT / BOOL / NULL / HASH / ARRAY
STRING       <- { '"' ( '\\' ([nrtv"] / [0-9]^3) / [^"\\] )* '"' }
INT          <- { [0-9]+ }
FLOAT        <- { [0-9]* '.' [0-9]+ }
BOOL         <- { 'true' / 'false' }
NULL         <- { 'null' }
CBOPEN       <- '{'
CBCLOSE      <- '}'
ABOPEN       <- '['
ABCLOSE      <- ']'
COMMA        <- ','
COLON        <- ':'
END          <- !.

\end{verbatim}
\end{myquote}
\textit{Example of a piece of Naigama grammar}

\subsection{Rules}

A rule is defined as an \textit{Identifier}, followed by a
left-pointing arrow (composed of a less-than and a minus sign),
followed by a matching expression.

When a Naigama grammar consists of a sequence of rules
(as opposed to a single line expression),
the first rule is used as the starting point for matching inputs.

\begin{myquote}
\begin{verbatim}
RULE1 <- 'a'
RULE2 <- 'b'
RULE3 <- 'c'

\end{verbatim}
\end{myquote}
\textit{Definition of a set of rules}

\subsubsection{Identifiers}

Identifiers, in Naigama, are defined as a combination of letters,
numbers and the underscore character, not starting with a number,
of between one and 64 characters long.

\begin{myquote}
\begin{verbatim}
IDENTIFIER <- [a-zA-Z_][a-zA-Z0-9_]^-63

\end{verbatim}
\end{myquote}
\textit{Definition of an identifier}

Identifiers are used to start rule definitions, and as references
to rules in expressions.

\subsubsection{Implicit Rules}

\subsection{Expressions}

Expressions are lists of terms, optionally separated by the
OR operator (denoted by the forward slash sign).

\subsection{Terms}

\subsection{Matchers}

\subsubsection{NOT and AND Modifiers}

\subsubsection{Quantifiers}

\subsubsection{The ANY Matcher}

The ANY matcher also allows you to define end-of-input, like so:

\begin{myquote}
\begin{verbatim}
ENDOFINPUT <- !.

\end{verbatim}
\end{myquote}
\textit{Definition of end-of-input}

\subsubsection{The SET Matcher}

\subsubsection{The STRING Matcher}

\subsubsection{The BITMASK Matcher}

\subsubsection{The HEXLITERAL Matcher}

\subsubsection{The VARCAPTURE Matcher}

\subsubsection{The CAPTURE Matcher}

\subsubsection{The GROUP Matcher}

\subsubsection{The MACRO Matcher}

\subsubsection{The VARREFERENCE Matcher}

\subsubsection{The REFERENCE Matcher}


