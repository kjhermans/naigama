\subsection{Instruction: noop}

\subsubsection{Mode}
This instruction is available in both mode 0 (parser) and mode 1 (script interpreter).
\subsubsection{Summary}

\InputIfFileExists{instr_noop_summary.tex}{}{}

\subsubsection{Grammar and Compiling}

\InputIfFileExists{instr_noop_compiling.tex}{}{}

\subsubsection{Assembly Syntax}

\begin{myquote}
\begin{verbatim}
NOOPINSTR <- { 'noop' }
\end{verbatim}
\end{myquote}

\subsubsection{Bytecode Encoding}

This instruction has a size of 4 bytes and is structured in bytecode as follows:

%DEADBEEF
$_{00}$\ 
\fbox{%
  \parbox{20pt}{%
00
  }%
}
\fbox{%
  \parbox{20pt}{%
00
  }%
}
\fbox{%
  \parbox{20pt}{%
00
  }%
}
\fbox{%
  \parbox{20pt}{%
00
  }%
}

%DEADBEEF
\InputIfFileExists{instr_noop_bytecode.tex}{}{}

\subsubsection{Execution State Change}

.

\InputIfFileExists{instr_noop_state.tex}{}{}

