\section{Conclusion}

[Threat \thethreatintent] is not addressed by Naigama. It is assumed that
the the grammar syntax is intuitive enough for the
person drafting policy. Otherwise, it is assumed that a GUI application
exists which can make this task easier.

[Threat \thethreatbcerror] is addressed by modularization of the Naigama
toolchain, allowing replacement, foregoing and reversing of modules and / or
their function. Debugging and fuzzing, both roadmap items, should make the
toolchain more complete.

[Threat \thethreatbcexec] is addressed by endless loop detection.

However, intuitively, it seems as if more could be done here.
Debugging could play a role, as well as several other, perhaps
more 'artificially intelligent' options (for example, when the stack
grows in size to far beyond the amount of bytes in input).

[Threat \thethreatbcsign] is not yet addressed by Naigama. This is in
the roadmap section (secure file format).

[Threat \thethreatbcupset] is addressed by the Hamming distance of
opcodes, the alignment of instructions, and the possibility to insert
trap instructions in the bytecode.

[Threat \thethreatengine] is only addressed by Naigama in the sense that
it intuitively seems to be well placed for implementation
in hardware.

[Threat \thethreatinput] is addressed by Naigama by strictly enforcing
the read-only-ness of bytecode and input.
