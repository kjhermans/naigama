\subsection{Instruction: call}

\subsubsection{Mode}
This instruction is available in mode 0 (parser).
\subsubsection{Summary}

\InputIfFileExists{instr_call_summary.tex}{}{}

\subsubsection{Grammar and Compiling}

\InputIfFileExists{instr_call_compiling.tex}{}{}

\subsubsection{Assembly Syntax}

\begin{myquote}
\begin{verbatim}
CALLINSTR <- { 'call' } S LABEL
S <- %s+
LABEL <- { [a-zA-Z0-9_]^1-64 }
\end{verbatim}
\end{myquote}

\subsubsection{Bytecode Encoding}

This instruction has a size of 8 bytes and is structured in bytecode as follows:

%DEADBEEF
$_{00}$\ 
\fbox{%
  \parbox{20pt}{%
00
  }%
}
\fbox{%
  \parbox{20pt}{%
04
  }%
}
\fbox{%
  \parbox{20pt}{%
03
  }%
}
\fbox{%
  \parbox{20pt}{%
82
  }%
}



$_{04}$\ 
\fbox{%
  \parbox{20pt}{%
00
  }%
}
\fbox{%
  \parbox{20pt}{%
00
  }%
}
\fbox{%
  \parbox{20pt}{%
00
  }%
}
\fbox{%
  \parbox{20pt}{%
00
  }%
}

%DEADBEEF
\InputIfFileExists{instr_call_bytecode.tex}{}{}

\subsubsection{Execution State Change}

.

\InputIfFileExists{instr_call_state.tex}{}{}

