\subsection{Instruction: call}

\subsubsection{Summary}

\subsubsection{Grammar and Compiling}

A 'call' instruction is emitted by the compiler when a reference is used
in an expression.

\subsubsection{Assembly Syntax}

\begin{myquote}
\begin{verbatim}
CALLINSTR <- 'call' S LABEL
S         <- %s+
LABEL     <- [a-zA-Z0-9_]^1-64

\end{verbatim}
\end{myquote}

\subsubsection{Bytecode Encoding}

This instruction is structured in bytecode as follows:

%DEADBEEF
$_0$\ 
\fbox{%
  \parbox{20pt}{%
00
  }%
}
\fbox{%
  \parbox{20pt}{%
04
  }%
}
\fbox{%
  \parbox{20pt}{%
03
  }%
}
\fbox{%
  \parbox{20pt}{%
33
  }%
}

%DEADBEEF

$_4$\
\fbox{%
  \parbox{20pt}{%
nn
  }%
}
\fbox{%
  \parbox{20pt}{%
nn
  }%
}
\fbox{%
  \parbox{20pt}{%
nn
  }%
}
\fbox{%
  \parbox{20pt}{%
nn
  }%
}

Where 'nn' is a 32-bit, network encoded, valid bytecode offset.

\subsubsection{Execution State Change}

.

Original state: \textit{(p, i, e, c)}

Operation: \textbf{call n}

Failure state: -

Success state: \textit{(n, i, e:(p+1), c)}

