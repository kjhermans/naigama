\section{Generations}

Naigama is written in generations. That is to say:
each subsequent generation of Naigama uses the tools generated
in the previous one, to build itself.
Initially, there's a bootstrap compiler, followed by compilers
that use their own grammar to parse their own grammer, followed
by one that can completely compile itself.

This leads to the following sequence:

\begin{itemize}

\item{Generation zero: This consists solely of a grammar compiler
      and assembler written in perl.}

\item{Generation one: This generation is the grammar only tooling
      built in C, based on the bytecode to parse grammar grammar
      and assembly grammar, generated by generation zero.}

\item{Generation two: equal to generation one, but then based on
      the bytecode generated by the tooling in generation one.}

\item{Generation three: native tooling, including scripting,
      based on the bytecode generated by the generation two tooling.}

\end{itemize}
