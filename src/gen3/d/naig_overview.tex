\section{Overview}

This document describes the Naigama parsing system, which is a 
modular system to process structured arbitrary length data inputs,
in order to extract meaning (matching, capturing from data), or to
manipulate (replacements within data).
The system was designed with a focus on information and system security.

The system comes in three parts: a grammar compiler, an assembler,
and a bytecode execution engine. Each of the parts has their own
input file format specification (grammar description, assembly specification,
and bytecode specification), and outputs the format of the next stage.
The process in brief: the compiler takes a
grammar description that the human end user has made, and turns it into
an assembly language. The assembler takes the assembly, and turns it
into a bytecode. The engine takes the bytecode and the input, and
processes it, resulting either in (various measures of) failure or
success.

%\begin{myquote}
%\begin{verbatim}
%Program:      _ Compiler       _  Assembler       _ Engine
%              /|         \     /|           \     /|  /|\  \
%             /           _\|  /             _\|  /     |   _\|
%Format: Grammar          Assembly           Bytecode   |   Output
%                                                       |
%                                                     Input
%\end{verbatim}
%\end{myquote}
%\textit{Schematic overview of Naigama data format transformation and modules}

\begin{center}
\begin{tikzpicture} [
bluenode/.style={rectangle, draw=blue!60, fill=green!5, very thick, minimum size=5mm},
rednode/.style={rectangle, draw=red!60, fill=red!5, very thick, minimum size=5mm},
]
%Nodes
\node[rednode]     (grammar)                            {Grammar};
\node[bluenode]    (compiler)    [right=of grammar]     {Compiler};
\node[rednode]     (assembly)    [below=of grammar]     {Assembly};
\node[bluenode]    (assembler)   [below=of compiler]    {Assembler};
\node[rednode]     (bytecode)    [below=of assembly]    {Bytecode};
\node[bluenode]    (engine)      [below=of assembler]   {Engine};
\node[rednode]     (input)       [below=of engine]      {Input};
\node[rednode]     (output)      [right=of engine]      {Output};

%Lines
\draw[->] (grammar.east) -- (compiler.west);
\draw[->] (compiler.south) -- (assembly.north);
\draw[->] (assembly.east) -- (assembler.west);
\draw[->] (assembler.south) -- (bytecode.north);
\draw[->] (bytecode.east) -- (engine.west);
\draw[->] (input.north) -- (engine.south);
\draw[->] (engine.east) -- (output.west);

\end{tikzpicture}

\textit{Schematic overview of Naigama data format transformation and tools}

\end{center}


The reason for this modularity is strict separation of tasks, and openness:
anyone should be able to take out a module, and replace it with one
of their own.
It should be expressly possible to take out the compiler, and replace
it with another tool that produces the Naigama assembly, for example.
Or have a different engine. Or write an optimizer for the assembly.

\subsection{On Grammar}
  
Naigama grammar is the human interface of the system.
It can be used to define complex syntax definitions in order
to match, capture from, and replace in, structured inputs.

People familiar with regular expressions \cite{bib:regex},
Backus-Naur syntax descriptions \cite{bib:backusnaur},
and / or Lex and Yacc tools \cite{bib:yacc},
should find this part of the document relatively easy to understand.
The ideas underlying Naigama grammar, assembly and bytecode
are heavily borrowed from Lua PEG (LPEG) \cite{bib:peg}
by Roberto Ierusalimschy et al.

\subsection{On Assembly}

The Naigama assembly language is a mixture of two languages:
For one, it is the outcome instruction set, according to the
aforementioned LPEG whitepaper,
of the Naigama grammar compilation process. However, it also contains
instructions that won't be produced by this compiler, but may
either be produced by a separate optimizer, or other compilers
altogether (for example, one that focuses on network packet
parsing).

The Naigama assembly language, in all cases, is the human readable
variant of the Naigama bytecode, and its instructions are one-on-one
translatable from the one to the other, and vice versa
(although labels will be lost in the reverse process).

\subsection{On Bytecode}

Naigama bytecode is the machine interface of the system.
It runs in the Naigama engine against the input provided by the user.
It is designed to be:

\begin{itemize}

\item Easy and quick to interpret.
\item Resilient against bit upset events.
\item Resilient against endless loops.
\item Usable while keeping both bytecode and input as read-only buffers.

\end{itemize}
