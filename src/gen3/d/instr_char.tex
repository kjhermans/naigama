\subsection{Instruction: char}

\subsubsection{Summary}

\InputIfFileExists{instr_char_summary.tex}{}{}

\subsubsection{Grammar and Compiling}

\InputIfFileExists{instr_char_compiling.tex}{}{}

\subsubsection{Assembly Syntax}

\begin{myquote}
\begin{verbatim}
CHARINSTR <- { 'char' } S HEXBYTE
S <- %s+
HEXBYTE <- { [0-9a-fA-F]^2 }
\end{verbatim}
\end{myquote}

\InputIfFileExists{instr_char_assembly.tex}{}{}

\subsubsection{Bytecode Encoding}

This instruction has a size of 8 bytes and is structured in bytecode as follows:

%DEADBEEF
$_{00}$\ 
\fbox{%
  \parbox{20pt}{%
00
  }%
}
\fbox{%
  \parbox{20pt}{%
04
  }%
}
\fbox{%
  \parbox{20pt}{%
03
  }%
}
\fbox{%
  \parbox{20pt}{%
d7
  }%
}



$_{04}$\ 
\fbox{%
  \parbox{20pt}{%
00
  }%
}
\fbox{%
  \parbox{20pt}{%
00
  }%
}
\fbox{%
  \parbox{20pt}{%
00
  }%
}
\fbox{%
  \parbox{20pt}{%
00
  }%
}

%DEADBEEF
\InputIfFileExists{instr_char_bytecode.tex}{}{}

\subsubsection{Execution State Change}

.

\InputIfFileExists{instr_char_state.tex}{}{}

