\subsection{Instruction: testchar}

\subsubsection{Summary}

\InputIfFileExists{instr_testchar_summary.tex}{}{}

\subsubsection{Grammar and Compiling}

\InputIfFileExists{instr_testchar_compiling.tex}{}{}

\subsubsection{Assembly Syntax}

\begin{myquote}
\begin{verbatim}
TESTCHARINSTR <- { 'testchar' } S HEXBYTE S LABEL ( S AMPERSAND HEXBYTE )?
S <- %s+
HEXBYTE <- { [0-9a-fA-F]^2 }
LABEL <- { [a-zA-Z0-9_]^1-256 }
AMPERSAND <- '&'
\end{verbatim}
\end{myquote}

\subsubsection{Bytecode Encoding}

This instruction has a size of 12 bytes and is structured in bytecode as follows:

%DEADBEEF
$_{00}$\ 
\fbox{%
  \parbox{20pt}{%
00
  }%
}
\fbox{%
  \parbox{20pt}{%
08
  }%
}
\fbox{%
  \parbox{20pt}{%
03
  }%
}
\fbox{%
  \parbox{20pt}{%
9a
  }%
}



$_{04}$\ 
\fbox{%
  \parbox{20pt}{%
00
  }%
}
\fbox{%
  \parbox{20pt}{%
00
  }%
}
\fbox{%
  \parbox{20pt}{%
00
  }%
}
\fbox{%
  \parbox{20pt}{%
00
  }%
}



$_{08}$\ 
\fbox{%
  \parbox{20pt}{%
00
  }%
}
\fbox{%
  \parbox{20pt}{%
00
  }%
}
\fbox{%
  \parbox{20pt}{%
00
  }%
}
\fbox{%
  \parbox{20pt}{%
00
  }%
}

%DEADBEEF
\InputIfFileExists{instr_testchar_bytecode.tex}{}{}

\subsubsection{Execution State Change}

.

\InputIfFileExists{instr_testchar_state.tex}{}{}

