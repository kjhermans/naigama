\section{Counters}

An addition that the Naigama makes to LPEG, is the introduction
of counters to prevent matchers being written out completely
when a large, but absolute, upper bound of a matcher's quantifier
is given.

For example:

\begin{myquote}
\begin{verbatim}
HEXLITERAL <- [0-9a-fA-F]^8-32

\end{verbatim}
\end{myquote}

Must be written out by LPEG as repetitions, like so:

\begin{myquote}
\begin{verbatim}
  set 000000000000ff037e0000007e000000000000000000000000000 ...
  set 000000000000ff037e0000007e000000000000000000000000000 ...
  set 000000000000ff037e0000007e000000000000000000000000000 ...
-- ... repeated eight times total ...

  set 000000000000ff037e0000007e000000000000000000000000000 ...
  partialcommit LABEL1
LABEL1:
-- ... repeated 24 times ...

\end{verbatim}
\end{myquote}

Naigama solves this by introducing counters. These are numbered
registers for a counter, whose value can only go down (to zero).
The instructions to implement a counted loop are
'counter' and 'condjump'. The first one sets the counter's value,
the second one only jumps when the value is non zero.
The grammar above is then compiled as follows:

\begin{myquote}
\begin{verbatim}
  call HEXLITERAL
  end
-- Rule
HEXLITERAL:
  counter 0 8
__TERM_2:
  set 000000000000ff037e0000007e000000000000000000000000000 ...
  condjump 0 __TERM_2
  catch __TERM_3
  counter 1 24
__TERM_4:
  set 000000000000ff037e0000007e000000000000000000000000000 ...
  partialcommit __TERM_5
__TERM_5:
  condjump 1 __TERM_4
  commit __TERM_3
__TERM_3:
  ret

\end{verbatim}
\end{myquote}

Not only delivers this feature much more efficient code, the code
is also less cluttered, and more de-compileable.
