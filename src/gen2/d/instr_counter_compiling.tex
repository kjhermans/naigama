This instruction is emitted in conjunction with a 'condjump'
instruction, when a matcher quantifier is used.
For example:

\begin{myquote}
\begin{verbatim}
RULE <- 'a'^5
\end{verbatim}
\end{myquote}

Will wrap the matcher bytecode between a counter and a condjump
instruction, initial value of the counter of which will be five.
Typically, as follows (pattern):

\begin{myquote}
\textit{counter 0 5} \newline
\textit{jumpbacklabel:} \newline
\textit{[[matcher]]} \newline
\textit{condjump 0 jumpbacklabel} \newline
\end{myquote}

The 'condjump' instruction will either jump back (when the counter
value in the register has not yet reached zero), or ignore the
jump and move on (when it has). Thus implementing a loop.
