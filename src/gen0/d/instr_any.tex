\subsection{Instruction: any}

\subsubsection{Summary}

The 'any' instruction checks to see if there is a character of input
left to consume, and consumes it if there is. Otherwise it causes a failure.

\subsubsection{Grammar and Compiling}

The '.' (dot) matcher used in grammar, like in regular expressions
\cite{bib:regex}, means 'match any character'. In Perl \cite{bib:perl}
a modifier has to be added to make it include matching of newlines.
In Naigama, 'any character' is confined to 8-bit characters (bytes).

\subsubsection{Assembly Syntax}

\begin{myquote}
\begin{verbatim}
ANYINSTR <- 'any'

\end{verbatim}
\end{myquote}

\subsubsection{Bytecode Encoding}

This instruction is structured in bytecode as follows:

%DEADBEEF
$_0$\ 
\fbox{%
  \parbox{20pt}{%
00
  }%
}
\fbox{%
  \parbox{20pt}{%
00
  }%
}
\fbox{%
  \parbox{20pt}{%
03
  }%
}
\fbox{%
  \parbox{20pt}{%
21
  }%
}

%DEADBEEF
\subsubsection{Execution State Change}

.

Original state: \textit{(p, i, e, c)}

Operation: \textbf{any} ; i \ \textless \ $\vert$S$\vert$

Failure state: \textit{(\textbf{Fail}, i, e, c)}

Success state: \textit{(p + 1, i + 1, e, c)}

