\section{Assembly}
\label{sec:assembly}

A Naigama assembly file or buffer consists of sequences of
whitespace, comments, labels, and (parametrized) instructions, separated
by new lines, and denoted in ASCII text.

Naigama assembly is taken by the Naigama assembler program (naia) and turned
into bytecode.

\begin{myquote}
\begin{verbatim}
-- Compilation of: TEST <- { 'a' } { 'a' } { 'a' / 'b' }
  call TEST
  end
-- Rule
TEST:
  opencapture 0
  char 61
  closecapture 0 0
  opencapture 1
  char 61
  closecapture 1 0
  opencapture 2
  catch __LABEL_30 -- alternative
  char 61
  commit __LABEL_31
__LABEL_30:
  char 62
__LABEL_31:
  closecapture 2 0
  ret
\end{verbatim}
\end{myquote}
\textit{Example of a piece of Naigama assembly}

\subsection{Comments}

Just like in Naigama grammar,
comments in Naigama assembly start with two minus signs, and end at
a new line. They can be given on separate lines, or they can be
postfixed to instructions or labels.

\subsection{Labels}

Labels are identifiers (the same identifiers as in their definition
in the grammar chapter) or numbers, followed by a colon sign. They are used
as positions for the bytecode to jump to, when used by instructions.

When performing the assembly, the Naigama assembler program resolves
each label to their offset in the resultant bytecode, and replaces each
reference to a label with that offset.

Labels don't quite disappear though - you have the option to make the
assembler program emit a so called 'labelmap' file, which stores the old
label names, mapped to the bytecode offsets given by the assembler,
and which can be used later
for debugging purposes by the bytecode execution engine.

When disassembling (bytecode to assembly), each instruction is always
prefixed by a label in the form of the instruction's offset in decimal,
so that jumps are always correct (but non descriptive).

\subsection{Instructions}


%\begin{table}[]
\begin{center}
\caption{Naigama Assembly Instructions}
\label{tab:naig_assembly}
\begin{longtable}{lll}
\textbf{Mnemonic} & \textbf{Param1} & \textbf{Param2} \\
\endhead
'any' &  &  \\
'backcommit' &  & LABEL \\
'call' &  & LABEL \\
'catch' &  & LABEL \\
'char' & char &  \\
'closecapture' & slot &  \\
'commit' &  & LABEL \\
'condjump' & register & LABEL \\
'counter' & register & value \\
'end' & code &  \\
'endisolate' &  &  \\
'endreplace' &  &  \\
'fail' &  &  \\
'failtwice' &  &  \\
'intrpcapture' &  &  \\
'isolate' & slot &  \\
'jump' &  & LABEL \\
'maskedchar' & char & mask \\
'noop' &  &  \\
'opencapture' & slot &  \\
'partialcommit' &  & LABEL \\
'quad' & quad &  \\
'range' & from & until \\
'replace' & slot & LABEL \\
'ret' &  &  \\
'set' & set &  \\
'skip' & number &  \\
'span' & set &  \\
'testany' &  & LABEL \\
'testchar' & char & LABEL \\
'testquad' & quad & LABEL \\
'testset' & set & LABEL \\
'trap' &  &  \\
'var' & slot &  \\
\end{longtable}
\end{center}
%\end{table}


See table [\ref{tab:naig_assembly}].

\subsection{Parameters}

Parameters follow the instruction in assembly text. They are separated
from the instruction and any other parameters by spaces. The following
types of parameters exist:

\subsubsection{Characters}

Character parameters are denoted as two byte hexadecimal values.

\subsubsection{Quads}

Quad parameters are denoted as eight byte hexadecimal values.

\subsubsection{Sets}

Set parameters are denoted as 64 byte hexadecimal values, representing
a bitmask of 256 possible booleans.

\subsubsection{Registers, Slots, Codes and Numbers}

These parameters are all denoted as decimal numbers.
