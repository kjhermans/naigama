\subsection{Instruction: intrpcapture}

\subsubsection{Summary}

\InputIfFileExists{instr_intrpcapture_summary.tex}{}{}

\subsubsection{Grammar and Compiling}

\InputIfFileExists{instr_intrpcapture_compiling.tex}{}{}

\subsubsection{Assembly Syntax}

\begin{myquote}
\begin{verbatim}
INTRPCAPTUREINSTR <- { 'intrpcapture' } S INTRPCAPTURETYPES
S <- %s+
INTRPCAPTURETYPES <- { 'ruint32' }
\end{verbatim}
\end{myquote}

\subsubsection{Bytecode Encoding}

This instruction has a size of 12 bytes and is structured in bytecode as follows:

%DEADBEEF
$_{00}$\ 
\fbox{%
  \parbox{20pt}{%
00
  }%
}
\fbox{%
  \parbox{20pt}{%
08
  }%
}
\fbox{%
  \parbox{20pt}{%
00
  }%
}
\fbox{%
  \parbox{20pt}{%
0f
  }%
}



$_{04}$\ 
\fbox{%
  \parbox{20pt}{%
00
  }%
}
\fbox{%
  \parbox{20pt}{%
00
  }%
}
\fbox{%
  \parbox{20pt}{%
00
  }%
}
\fbox{%
  \parbox{20pt}{%
00
  }%
}



$_{08}$\ 
\fbox{%
  \parbox{20pt}{%
00
  }%
}
\fbox{%
  \parbox{20pt}{%
00
  }%
}
\fbox{%
  \parbox{20pt}{%
00
  }%
}
\fbox{%
  \parbox{20pt}{%
00
  }%
}

%DEADBEEF
\InputIfFileExists{instr_intrpcapture_bytecode.tex}{}{}

\subsubsection{Execution State Change}

.

\InputIfFileExists{instr_intrpcapture_state.tex}{}{}

