\section{Binary Formats}

This section describes the formats used by the Naigama tooling, other than
the ones described in the sections on grammar, assembly and bytecode.

\subsection{Slotmap Format}

The slotmap file exists to help developers by easing access to
captures in complex grammars.

The Naigama grammar compiler can be instructed to emit a slotmap file.
This is a file which maps a unique name to a capture region's index.
This is provided so that, instead of using the index of capture region
(which requires hand counting them in your grammar file, which can be
tedious and error-prone, and something that would not survive
a grammar reshuffle, or the introduction of a capture region before
the one you're interested in), you can use a naming scheme for your
capture regions.

Names in the slotmap file are bound semantically to the capture region:
they are made up of the name of the rule,
an underscore, and all alphabetic characters in the capture region,
cast to uppercase. Should any name occur twice, it will be postfixed with
a counter until it's unique.

For example, the following rule with capture region definition:

\begin{myquote}
\begin{verbatim}
RULE <- { IDENT } OPTARGS LEFTARROW EXPRESSION
\end{verbatim}
\end{myquote}

will result in slotmap identifier 'RULE\_IDENT'.

The binary format of a slotmap file is composed as a sequence of binary records:
the slot index, denoted as a 32 bit network order unsigned integer, a
field of 32 bit all ones, and the slot name, denoted as a zero-terminated
string.

\begin{myquote}
\begin{verbatim}
00 00 00 00 ff ff ff ff 52 55 4c 45 5f 49 44 45  ........RULE_IDE
4e 54 00 00 00 00 01 ff ff ff ff 45 58 50 52 45  NT.........EXPRE
53 53 49 4f 4e 5f 54 45 52 4d 53 00 00 00 00 02  SSION_TERMS.....
ff ff ff ff 45 58 50 52 45 53 53 49 4f 4e 5f 54  ....EXPRESSION_T
45 52 4d 53 5f 31 00 00 00 00 03 ff ff ff ff 45  ERMS_1.........E
58 50 52 45 53 53 49 4f 4e 5f 54 45 52 4d 53 5f  XPRESSION_TERMS_
32 00 00 00 00 04 ff ff ff ff 54 45 52 4d 53 5f  2.........TERMS_
54 45 52 4d 00 00 00 00 05 ff ff ff ff 54 45 52  TERM.........TER
4d 5f 4e 4f 54 41 4e 44 00 00 00 00 06 ff ff ff  M_NOTAND........

\end{verbatim}
\end{myquote}
\textit{Example of the head of a slotmap file, hex dumped}

\subsection{Labelmap Format}

The labelmap format is optionally emitted by the assembler, and exists to:

\begin{itemize}
\item Allow for better debugging, because offsets in bytecode can be reduced
      to more intuitively named sections (rule names always make it to
      the labelmap unchanged).
\item Allow for calling the bytecode at symbols directly, which in turn
      allows you to use a bytecode blob more as a database of functions.
\end{itemize}

The binary format of the labelmap file is composed as a sequence of
binary records: the bytecode offset, denoted as a 32 bit network order
unsigned integer, followed by the labelstring, followed by a zero byte.

\begin{myquote}
\begin{verbatim}
00 00 00 10 47 52 41 4d 4d 41 52 00 00 00 00 28  ....GRAMMAR....(
5f 5f 4c 41 42 45 4c 5f 31 31 00 00 00 00 38 5f  __LABEL_11....8_
5f 4c 41 42 45 4c 5f 31 32 00 00 00 00 40 5f 5f  _LABEL_12....@__
4c 41 42 45 4c 5f 36 00 00 00 00 48 5f 5f 4c 41  LABEL_6....H__LA
42 45 4c 5f 37 00 00 00 00 54 5f 5f 70 72 65 66  BEL_7....T__pref
69 78 00 00 00 00 5c 5f 5f 4c 41 42 45 4c 5f 32  ix....\__LABEL_2
39 00 00 00 00 7c 5f 5f 4c 41 42 45 4c 5f 34 33  9....|__LABEL_43
00 00 00 00 a8 5f 5f 4c 41 42 45 4c 5f 34 34 00  .....__LABEL_44.
00 00 00 b8 5f 5f 4c 41 42 45 4c 5f 33 32 00 00  ....__LABEL_32..
00 00 e4 5f 5f 4c 41 42 45 4c 5f 35 35 00 00 00  ...__LABEL_55...
01 10 5f 5f 4c 41 42 45 4c 5f 35 36 00 00 00 01  ..__LABEL_56....
10 5f 5f 4c 41 42 45 4c 5f 33 33 00 00 00 01 18  .__LABEL_33.....
5f 5f 4c 41 42 45 4c 5f 33 30 00 00 00 01 1c 45  __LABEL_30.....E
4e 44 00 00 00 01 34 5f 5f 4c 41 42 45 4c 5f 35  ND....4__LABEL_5
38 00 00 00 01 38 44 45 46 49 4e 49 54 49 4f 4e  8....8DEFINITION
00 00 00 01 4c 53 49 4e 47 4c 45 5f 45 58 50 52  ....LSINGLE_EXPR
45 53 53 49 4f 4e 00 00 00 01 60 52 55 4c 45 00  ESSION....`RULE.
00 00 01 a0 45 58 50 52 45 53 53 49 4f 4e 00 00  ....EXPRESSION..
00 01 e4 5f 5f 4c 41 42 45 4c 5f 31 30 36 00 00  ...__LABEL_106..

\end{verbatim}
\end{myquote}
\textit{Example of a section of labelmap file, hex dumped}

\subsection{Engine Output Format}

The Naigama bytecode execution engine, naie, produces, on success, output for
digital processing, in the form of a binary table, which is structured
as follows:

\begin{itemize}

\item Each record is four 32 bit integers, in network order.

\item The first record contains the end code of the matching process.
This is the same code as was given as a parameter to the 'end'
instruction that resulted in the execution finishing. This is the
first field, by default it is zero.
The second field contains the amount of subsequent records.

\item Subsequent records have as fields, either:

\begin{itemize}
\item Type, slot, start, stop (when of 'capture' type), or:
\item Type, zero, start, length (when of 'replace' type).
\end{itemize}

\end{itemize}

Bear in mind the following:

\begin{itemize}
\item 'Capture' type is denoted as 1, 'Replace' as 3.
\item Offsets and lengths of captures refer to the input buffer.
\item Offsets and lengths of replacements refer to the bytecode.
\end{itemize}

\begin{myquote}
\begin{verbatim}
00 00 00 00 00 00 03 64 00 00 00 00 00 00 00 00  .......d........
00 00 00 01 00 00 00 00 00 00 00 1c 00 00 00 23  ...............#
00 00 00 01 00 00 00 03 00 00 00 32 00 00 00 59  ...........2...Y
00 00 00 01 00 00 00 04 00 00 00 32 00 00 00 55  ...........2...U
00 00 00 01 00 00 00 14 00 00 00 32 00 00 00 55  ...........2...U
00 00 00 01 00 00 00 02 00 00 00 34 00 00 00 3f  ...........4...?
00 00 00 01 00 00 00 04 00 00 00 34 00 00 00 3f  ...........4...?
00 00 00 01 00 00 00 17 00 00 00 34 00 00 00 3e  ...........4...>
00 00 00 01 00 00 00 06 00 00 00 3e 00 00 00 3f  ...........>...?
00 00 00 01 00 00 00 03 00 00 00 42 00 00 00 53  ...........B...S
00 00 00 01 00 00 00 04 00 00 00 42 00 00 00 53  ...........B...S
00 00 00 01 00 00 00 17 00 00 00 42 00 00 00 53  ...........B...S

\end{verbatim}
\end{myquote}
\textit{Example of the head of engine output with a zero end code
and 868 matches (all captures), hex dumped}

