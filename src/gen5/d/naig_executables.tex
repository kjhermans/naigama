\section{Executables}

\subsection{The Compiler}

The Naigama grammar compiler is called 'naic'.
It takes a grammar file as input, and outputs assembly text.
It can be invoked as follows:

\begin{myquote}
\begin{verbatim}
$ naic -i myfile.niag -o myfile.asm
\end{verbatim}
\end{myquote}
\textit{Example of the invocation of the Naigama grammar compiler}

Bear in mind that both the input (grammar) file, as well as the
output (assembly) file may be omitted (in which case they will be
assumed to be stdin and stdout, respectively), or be denoted as
a minus sign ('-'), which will have the same effect.

You may also use the following options:

\begin{itemize}
\item \texttt{-m $<$path$>$} Tells the compiler to emit the slotmap
      file in $<$path$>$.
\item \texttt{-D} Creates a lot of debugging output.
\item \texttt{-t} Tells the compiler to surround generated rule
      code with 'trap' instructions.
\end{itemize}

\subsection{The Assembler}

The Naigama grammar assember is called 'naia'.
It takes an assembly file as input, and outputs bytecode.
It can be invoked as follows:

\begin{myquote}
\begin{verbatim}
$ naia -i myfile.asm -o myfile.byc
\end{verbatim}
\end{myquote}
\textit{Example of the invocation of the Naigama assembler}

Bear in mind that both the input (assembly) file, as well as the
output (bytecode) file may be omitted (in which case they will be
assumed to be stdin and stdout, respectively), or be denoted as
a minus sign ('-'), which will have the same effect.

You may also use the following options:

\begin{itemize}
\item \texttt{-l $<$path$>$} Tells the assembler to emit the labelmap
      file in $<$path$>$.
\item \texttt{-D} Creates a lot of debugging output.
\end{itemize}

\subsection{The Engine}

The Naigama bytecode execution engine is called 'naie'.
It takes a bytecode file as input, as well as an input file.
It can be invoked as follows:

\begin{myquote}
\begin{verbatim}
$ naie -c myfile.byc -i myfile.dat -o myfile.out
\end{verbatim}
\end{myquote}
\textit{Example of the invocation of the Naigama engine}

\subsection{The Disassembler}

The Naigama disassembler is called 'naid'.
It takes a bytecode file as input, and outputs assembly.

\begin{myquote}
\begin{verbatim}
$ naid -i myfile.byc -o myfile.asm
\end{verbatim}
\end{myquote}
\textit{Example of the invocation of the Naigama disassembler}

The output of the disassembler will differ from the assembly
generated by the compiler in that:

\begin{itemize}
\item Textual labels will be gone, instead:
\item Every instruction will be prefixed by a numeric label which
      is identical to that instruction's position offset in the bytecode, and
\item All jumps will therefore also be using those position labels.
\end{itemize}
