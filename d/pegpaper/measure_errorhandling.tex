\subsection{Error Handling}

Naigama knows three types of error condition:

\begin{itemize}
\item No match: a FAIL condition was encountered, and there was no 'catch' stack
element to divert the bytecode offset to an alternative path of execution,
rolling up the entire stack.
\item Bytecode error: an endless loop or trap instruction was encountered.
The latter is intended to catch errors of the third type (bytecode corruption),
but the engine cannot know that it has not been intentionally directed to
encounter this instruction.
\item Bytecode corruption: misalignment, bad instruction, pointers outside
the bytecode memory area all lead to immediate halting of execution.
\end{itemize}
