\section{Bytecode}


\begin{table}[]
\centering
\caption{Naigama Bytecode Instructions}
\label{tab:naig_bytecode}
\begin{tabular}{lllll}
\textbf{Mnemonic} & \textbf{Opcode} & \textbf{Param1} & \textbf{Param2} & \textbf{Length} \\
\end{tabular}
\end{table}


\subsection{Bytecode Structure}

A Naigama bytecode file or buffer consists of a sequence of binary
instructions which, in turn, each consist of a binary
encoded opcode, plus their parameters, should they have any.

The amount and kind of parameters following an opcode, is strictly
defined:
the same opcode will always be followed by the same kinds of parameters
and therefore, an instruction type will always be the same size
(see table [\ref{tab:naig_bytecode}]).

Naigama bytecode is taken by the Naigama engine program (naie) or
library, and run against an input, to produce an output.

\subsection{Opcode Values}

Opcode values are determined through

\begin{itemize}
\item Grouping; 
\item Hamming distance;
\item Instruction size;
\end{itemize}

\subsection{Noop Slides and Canaries}

Implementations that want each instruction to have exactly the
same size, can choose to pad the encoding of shorter instructions
with either no-ops, or canaries.

\subsection{Encoding of Parameters}

\subsubsection{Address}

\subsubsection{Char}

\subsubsection{Slot}

\subsubsection{Register}
