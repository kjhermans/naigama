\subsection{Instruction: ret}

\subsubsection{Summary}

\InputIfFileExists{instr_ret_summary.tex}{}{}

\subsubsection{Grammar and Compiling}

\InputIfFileExists{instr_ret_compiling.tex}{}{}

\subsubsection{Assembly Syntax}

\begin{myquote}
\begin{verbatim}
RETINSTR <- { 'ret' }
\end{verbatim}
\end{myquote}

\InputIfFileExists{instr_ret_assembly.tex}{}{}

\subsubsection{Bytecode Encoding}

This instruction has a size of 4 bytes and is structured in bytecode as follows:

%DEADBEEF
$_{00}$\ 
\fbox{%
  \parbox{20pt}{%
00
  }%
}
\fbox{%
  \parbox{20pt}{%
00
  }%
}
\fbox{%
  \parbox{20pt}{%
03
  }%
}
\fbox{%
  \parbox{20pt}{%
a0
  }%
}

%DEADBEEF
\InputIfFileExists{instr_ret_bytecode.tex}{}{}

\subsubsection{Execution State Change}

.

\InputIfFileExists{instr_ret_state.tex}{}{}

