\textbf{Abstract}

Because of its complexity - nested binary TLV encoding and various
compression schemes -
parsing BER and BER-like formatted messages (such as X.509 certificates or
SNMP messages) is considered problematic. To my
knowledge, formal parsing of BER is not
addressed by available solutions. Instead, custom, hand
coded solutions or generated code prevail. This is a security risk from a
point of view of bugs or code proliferation. The changes proposed in the
paper extend Parsing Expression Grammar
(PEG), which is an unambiguous, top-down, recursive descent parsing algorithm.
This PEG extension enables it to parse BER formats, while retaining a
simple grammar definition, and requiring only minimal changes to the
resulting assembly and bytecode interpreter. This paper also shows that,
having implemented those changes, and using a grammar that closely mimics
its ASN.1 counterpart definition, it is possible to split up X.509 and SNMPv3
messages in their underlying parts. Other findings presented
are that this extension allows BER well-formedness checks, as well as ASN.1
definition checks, and that it allows content, such as digital
signatures, to be isolated (captured) from the input (allowing for a subsequent
cryptographic content integrity check). It shows how OIDs
can be broken up in their composing parts, and the way in which text
fields of binary formats can be text-parsed (\textit{eg} email address
validation). Further study is required with respect to potential machine
implementations of the PEG engine, and the representation of binary
fields such as INTEGERs and OIDs from the input capture list.
