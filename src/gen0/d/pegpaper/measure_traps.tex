\subsection{Measure: Traps}
\label{sec:traps}

\subsubsection{Addition of the 'Trap' Instruction}

One or more 'trap' instructions can be emitted by the compiler in places that,
under normal circumstances, would be unreachable. For example, straight
after the following possible instructions:

\begin{itemize}
\item{jump}
\item{ret}
\item{end}
\item{commit}
\item{partialcommit}
\item{backcommit}
\item{fail}
\item{failtwice}
\end{itemize}

The possible advantage of placing traps is that, when the value of the
bytecode offset in the engine is somehow subverted, it may land on a trap
and halt execution.

The addition of the 'trap' instruction addresses [Threat \thethreatbcupset].
