\section{Namespaces}

Consider a (perl / json / python / etc) hashtable or object literal
notation. Usually, they go something like this:

\begin{myquote}
\begin{verbatim}
my $hash = {
  foo => 'bar',
  fubar => 'fu'
};

\end{verbatim}
\end{myquote}

The logic is similar also to XML-child entities: key-value pairs.
A value may be any object, even complex ones, where keys are
usually strings (or integers), and 'must' be unique. The penalty
for using the same key multiple times depends on the platform in
question: some throw an error, some stick to the first key given,
some to the last.

I would like to do namespace checking in the parser already.
As follows:

\begin{myquote}
\begin{verbatim}
CHILDREN <- [[ CHILD+ ]] -- namespace delimiter
CHILD    <- NAME ':' VALUE
NAME     <- { 'foo' }#1-3 / { 'bar' }#0- / { 'foobar' }#-1000
VALUE    <- int / string / boolean

\end{verbatim}
\end{myquote}


