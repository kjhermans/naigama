\section{Generations}

Naigama is written in generations. That is to say:
each subsequent generation of Naigama uses the tools generated
in the previous one, to build itself ('the grammar parses the grammar').

The following generations exist, with the following functions:

\begin{itemize}

\item Generation zero: This generation consists solely of a bootstrap
      grammar compiler and assembler written in perl.

\item Generation one: This generation is
      built in C, based on the bytecode to parse grammar grammar
      and assembly grammar, generated by generation zero.

\item Generation two: equal to generation one, but then based on
      the bytecode generated by the tooling in generation one.
      Generation two is 'pure' in that it has compiled itself,
      and it only provides parsing.

\item Generation three: This generation extends the concept of the
      language further in that it also allows scripting.
      The parser of the scripting language is not 'active' however,
      because it's performed by generation two.
      It's equivalent to generation zero, in that it is the bootstrap
      compiler for scripting.

\item Generation four: This generation has the compiler and assembler
      defined as active scripts, including grammar. There is no
      support required for their execution, save for the engine's
      implementation in C.

\item Generation five: This generation is 'pure' again, in that it
      is equivalent to generation four, but then compiled by itself.

\end{itemize}
