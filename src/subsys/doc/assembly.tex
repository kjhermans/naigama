\section{X}

\subsection{The Machine Concept}

\subsubsection{Registers}

\subsubsection{The FAIL Section}

The FAIL section is a two-register area that the engine jumps to
when comparison instructions produce false. The first register is
assumed to be an opcode, the (optional) second its (only) parameter.

\subsection{The FAIL Region}

The FAIL region is the standard piece of code prefixed to any output,
that implements failtwice and failure in general (whether explicit or
implicit). It is always this:

\begin{verbatim}
  JUMP <#L3>
#failtwice:
#L4:
  FAIL RAISE, end
  POP REG1
  FAIL JUMP, <#L5>
  CMP REG1, 3
  JUMP <#L6>
#L5:
  FAIL RAISE, fatal
  CMP REG1, 1
  POP REG1
  JUMP <#L4>
#L6:
#fail:
#L1:
  FAIL RAISE, end
  POP REG1
  FAIL JUMP, <#L2>
  CMP REG1, 3
  POP REG1
  POP REG2
  POP REG3
  ASSIGN actionlen, REG2
  ASSIGN inputpos, REG3
  JUMPREF REG1
#L2:
  FAIL RAISE, fatal
  CMP REG1, 1
  POP REG1
  JUMP <#L1>
#L3:

\end{verbatim}

\subsection{Instructions}

\subsubsection{Labeling}

\subsubsection{FAIL}

\subsubsection{JUMP}

\subsubsection{RAISE}

\subsubsection{PUSH}

\subsubsection{POP}

\subsubsection{LT}

\subsubsection{CMP}
