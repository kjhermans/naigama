\section{API's}

\subsection{Languages}

\subsection{C API Concepts}

\subsubsection{Main Structures}

\subsubsection{Error Handling}

Naigama C library function prototypes are (almost) all typed to return
the NAIG\_ERR\_T type.

\subsection{Compiler}

\subsection{Assembler}

\subsection{Engine}

\subsection{All in one API}

The compound API provides an interface to all of the action in one place.

\subsubsection{Includes}

\begin{myquote}
\begin{verbatim}
#include <naigama/naigama.h>
\end{verbatim}
\end{myquote}

\subsubsection{Initialization}

\subsubsection{Using the parser}

\begin{myquote}
\begin{verbatim}
NAIG_ERR_T naig_compile
  (naig_t* naig, char* grammar, int traps);
\end{verbatim}
\end{myquote}

\begin{myquote}
\begin{verbatim}
NAIG_ERR_T naig_run
  (
    naig_t*         naig,
    unsigned char*  input,
    unsigned        input_length,
    naio_result_t*  result
  );
\end{verbatim}
\end{myquote}

\subsubsection{Using the Result Structure}

To include the proper types and functions, use:

\begin{myquote}
\begin{verbatim}
#include <naigama/memio/naio.h>
\end{verbatim}
\end{myquote}

The definition of the associated types:

\begin{myquote}
\begin{verbatim}
typedef struct {
  uint32_t              action;
  uint32_t              slot;
  uint32_t              start;
  uint32_t              length; // dubs as char/quad in replace
}
naio_resact_t;

typedef struct
{
  int                   code;
  naio_resact_t*        actions;
  unsigned              length; // allocated
  unsigned              count;  // used
}
naio_result_t;
\end{verbatim}
\end{myquote}

You can step through each action (up to result-$>$count),
and for each result-$>$actions[ i ], check its action
(which can be one of NAIG\_ACTION\_OPENCAPTURE,
NAIG\_ACTION\_DELETE, NAIG\_ACTION\_REPLACE\_CHAR, or
NAIG\_ACTION\_REPLACE\_QUAD),
its slot number (this will be the iterator that the compiler uses
for each encountered capture region), and its start and length.
You'll still need the input buffer handy, because this type does
not provide you with the capture data itself.

If you're only interested in capturing, then you'll only
need the NAIG\_ACTION\_OPENCAPTURE action type. Also when
your grammar does not contain any replacement definitions, you'll never
encounter any other action type in the capture list.

\subsubsection{Performing Replacements}

...

\subsubsection{Getting a Malloced Capture Tree}

To handle capture data independently of the input buffer, and
preprocessed to represent a capture tree, you can use the following API:

\begin{myquote}
\begin{verbatim}
extern
naio_resobj_t* naio_result_object
  (
    const unsigned char*  input,
    unsigned              inputlength,
    naio_result_t*        result
  )
  __attribute__ ((warn_unused_result));
\end{verbatim}
\end{myquote}

\begin{myquote}
\begin{verbatim}
struct naio_resobj
{
  unsigned              type;
  char*                 string;
  unsigned              stringlen;
  unsigned              origoffset;
  naio_resobj_t*        parent;
  naio_resobj_t**       children;
  unsigned              nchildren;
  unsigned              slotnumber;
  void*                 auxptr; /* this one is for you */
};
\end{verbatim}
\end{myquote}

Below is a simplified version of the naio\_result\_object\_debug function,
which shows you how you can recursively step through your capture
list and print an indented capture result tree.

\begin{myquote}
\begin{verbatim}
void naio_result_object_debug
  (naio_resobj_t* object, unsigned indent)
{
  unsigned i;

  if ((int)(object->type) == -1) {
    fprintf(stderr, "TOP  - ");
  } else {
    fprintf(stderr, "%.4u ", object->type);
  }
  for (i=0; i < indent; i++) {
    fprintf(stderr, "--");
  }
  fprintf(stderr, "| %s\n", object->string);
  for (i=0; i < object->nchildren; i++) {
    naio_result_object_debug(object->children[ i ], indent + 1);
  }
}
\end{verbatim}
\end{myquote}

