\subsection{Instruction: replace}

\subsubsection{Summary}

Introduces a replace section in bytecode. This section consists only
of certain instructions, ends with a 'endreplace' instruction, and
can be jumped over, when replacing is not what the engine is configured
to do.

\subsubsection{Grammar and Compiling}

The 'replace' and 'endreplace' instructions are emitted by the compiler
when encountering a replace construct, which consists of a 'right arrow'
(a minus sign, followed by a greater than sign), followed by a list of
terms that may only be of type string or variable references, and run
until the end of the rule.

\begin{myquote}
\begin{verbatim}
CSVFIELD <- ( {:field: [^,] } ','? )* -> '<value>' $0 '</value>'

\end{verbatim}
\end{myquote}
\textit{(Crude) example of a replacement definition, turning 'csv' into 'xml'}

\subsubsection{Assembly Syntax}

\begin{myquote}
\begin{verbatim}
REPLACEINSTR <- 'replace' S LABEL S LABEL
S            <- %s+
LABEL        <- [a-zA-Z0-9_]^1-64

\end{verbatim}
\end{myquote}
\subsubsection{Bytecode Encoding}

This instruction is structured in bytecode as follows:

%DEADBEEF
$_0$\ 
\fbox{%
  \parbox{20pt}{%
00
  }%
}
\fbox{%
  \parbox{20pt}{%
08
  }%
}
\fbox{%
  \parbox{20pt}{%
03
  }%
}
\fbox{%
  \parbox{20pt}{%
7b
  }%
}

%DEADBEEF

$_4$\ 
\fbox{%
  \parbox{20pt}{%
nn
  }%
}
\fbox{%
  \parbox{20pt}{%
nn
  }%
}
\fbox{%
  \parbox{20pt}{%
nn
  }%
}
\fbox{%
  \parbox{20pt}{%
nn
  }%
}

$_8$\ 
\fbox{%
  \parbox{20pt}{%
mm
  }%
}
\fbox{%
  \parbox{20pt}{%
mm
  }%
}
\fbox{%
  \parbox{20pt}{%
mm
  }%
}
\fbox{%
  \parbox{20pt}{%
mm
  }%
}

Where 'nn' is the 32 bit network order encoding of the slot number
associated with the replacement, and 'mm' is the bytecode offset
to jump to, if the replacement section is to be skipped.

\subsubsection{Execution State Change}

.

Original state: \textit{(p, i, e, c)}

Operation: \textbf{any} ; i \ \textless \ $\vert$S$\vert$

Failure state: \textit{(\textbf{Fail}, i, e, c)}

Success state: \textit{(p + 1, i + 1, e, c)}

