\section{Easy Pickings and Assumptions}

\subsection{Coveying the Intention of the Grammar Writer}

Wrt [Threat \thethreatintent], it is assumed that the grammar writer has been
or can be taught to use Regular Expressions \cite{bib:regex}, and
Backus-Naur (-like) notation \cite{bib:backusnaur}. These are the fundamentals
behind PEG grammar notation when it comes to defining rule terminals and
non terminals respectively.

It is also assumed, should this not be the case, that for example, a GUI
based policy definition application, exists or can be made. This document
does not deal with the specifics of those.

\subsection{The Hardness of the Bytecode Execution Engine}

Wrt [Threat \thethreatengine], it is assumed that a bytecode execution
engine can be ported to hardware (or a hardware description language, such
as VHDL \cite{bib:vhdl})  with relative ease. This is at least
intuitively supported by the following facts:

\begin{itemize}
\item The amount of bytecode instructions is relatively low (around thirty,
      many of which are only emitted by the optimizer, which is a step
      that can be skipped). See table [Table \ref{tab:naig_bytecode}]
      for all Naigama parser instructions.
\item The amount of memory manipulations that must be performed per
      bytecode instruction is relatively low. For reference, the Naigama
      bytecode engine main loop consists of about 400 lines of C code
      [src/gen2/lib/engine/naie\_engine\_loop.c].
\end{itemize}

Hardware execution of the PEG bytecode engine should provide some
confidence the risk of bitfaults or attacks on the engine logic itself, can
be considered minimal [citation needed].
