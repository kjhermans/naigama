\section{Appendix: Instructions}

In this section, each Naigama instruction is treated in detail in its own subsection.
In these subsections, an instruction is disected as follows:

\begin{itemize}
\item 'Summary'. There is a short summary of the function of the instruction.
\item 'Grammar and Compiling'. This subsection details how and why this
      instruction may be emitted by the compiler, given the grammar.
      This may include compilation patterns.
\item 'Assembly Syntax'. This subsection provides the exact syntax of
      the instruction in Naigama assembly.
\item 'Bytecode Encoding'. This subsection provides a byte-for-byte
      specification of how this instruction will be laid out in memory
      and on disk.
\item 'Execution State Change'. This subsection contains the formal
      description of the execution of the instruction.
\item 'PseudoCode'. This subsection contains a pseudo code implementation
      of the instruction.
\end{itemize}
